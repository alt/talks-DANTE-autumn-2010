In der TeX-Welt hat sich in den letzten Jahren vieles getan: Neue TeX-Maschinen (XeTeX, luaTeX), Formate (LaTeX3) und einzelner Pakete (z.B. fontspec, unicode-math) erleichtern viele alltägliche Aufgaben, die sonst oft mühsam sein können. Dazu zählt z.B. das Einbinden von Schriften, was dank OpenType-Schriften und Xe/luaTeX eine sehr einfache Sache geworden ist, für die keine Pakete benötigt wird.

Verbindet man diese Entwicklungen mit modernen Tastaturbelegungen (Neo) und Kodierungen (Unicode, utf8), kann TeX als mächtiges, noch einfacher zu bedienendes Satzsystem verwendet werden.

Dieser Vortrag stellt einige Möglichkeiten vor, den TeX-Alltag zu vereinfachen, sowie einige experimentelle Versuche, die mit unkonventionellen Ansätzen Schreibarbeit sparen und den TeX-Quellcode in großen Teilen besser lesbar machen können.