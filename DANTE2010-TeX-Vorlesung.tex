\documentclass[t]{beamer}
\input{\jobname_header}
\input{\jobname_beameranpassungen}
\begin{document}

\begin{frame}{\LaTeX-Vorlesung an der Uni Heidelberg}
• Vorlesung „Einführung in das Textsatzsystem \LaTeX“ (SoSe 2010)
• Veranstaltet von der Informatik, 2\,SWS
• 2\,ECTS-Punkte, benotet
•[] (übergreifende Kompetenzen)
• Bewertungsgrundlage: Übungszettel
• 170 Teilnehmer (unverbindlich) angemeldet, ca. 100 Scheine erworben
\•
\end{frame}

\section{Warum \LaTeX\ an der Uni?}
\begin{frame}{\LaTeX\ im universitären Umfeld}
• Formelsatz: Mathematik, Physik
• hochwertige Dokumente: Sprachwissenschaften
• als freie Software überall verfügbar
• etc. … 
\•
\end{frame}

\begin{frame}{Kontakt}
• nicht jeder Studienanfänger kennt \LaTeX\ überhaupt
• oft nur den Namen gehört, aber keine Kenntnisse
•[⇒] Interesse vorhanden, oft kein Ansprechpartner
• als Vorlesung, um kompetenten Ansprechpartner zu bieten
• langfristige Beschäftigung, um selbst Kompetenz ausbilden zu können
• Vorteil gegenüber Crash-Kursen: Tieferes Verständnis statt nur spezielles Auswendiglernen
• Ziel: Eigenständiges Verfassen umfangreicher (Abschluss-)Arbeiten
• Vermeiden von unverstandenem copy-paste Code, der Probleme bereiten kann
\•
\end{frame}

\begin{frame}{Warum kein Crash-Kurs?}
• oft angeboten: zwei, drei Wochenenden zum Erlernen
•[-] reicht zum Anwenden, aber nicht zum Verstehen von \LaTeX
• als wichtigstes Mittel für wissenschaftliche Publikationen sollte man verstehen, was es tut und warum
• häufigere Beschäftigung mit dem Stoff erhöht die Aufnahmefähigkeit
• Umgang mit \LaTeX\ im gewohnten Umfeld, nicht im Rechnerraum …
\•
\end{frame}

\begin{frame}{Warum bewertete Vorlesung?}
• Anerkennung der erbrachten Übungsleistung
• Erwerb typographischer Grundkenntnisse als anerkennbare Studienleistung
\•
\end{frame}

\begin{frame}{Offizielle Organisation}
• offizieller Verantwortlicher: Professor der Informatik
• zwei HiWis
\•
\end{frame}

\section{Inhalte der Vorlesung}
\begin{frame}[fragile]
• orientiert an modernen Maschinen
• vorgeschriebene Codierung: utf8 (Vermeiden von allen möglichen Problemen)
•[⇒] fast ausschließlich Verwendung von \XeTeX
• einheitlicher Editor (\TeX works)
• …
• \url{http://elearning.uni-heidelberg.de/mod/resource/view.php?id=87822}
\•
\end{frame}

\begin{frame}{\TeX nische Umsetzung: Folien}
• Ziel: einheitliches Layout für alle Folien
• jedes Dokument eigenständig
• Problem: Anzahl der Pakete beschränkt (Counter, Längen etc., Inkompatibilitäten)
• geringst möglicher Aufwand zum Setzen (automatisches Erkennen von Titel, Vorlesungsnummer, …)
\•
\end{frame}

\begin{frame}{\TeX nische Umsetzung: Folien}
• gewünscht: einheitliches Layout für alle Folien
• Verwendung von beamer
• eigene header-Datei
• eigene Datei für Anpassungen des Layouts
• experimentelle Methode, um den Titel aus dem Dateinamen zu extrahieren
\•
\end{frame}

\begin{frame}[fragile]{Header der einzelnen Vorlesungen}
\begin{verbatim}
\input{kursheader}
\begin{document}
\end{verbatim}
\end{frame}

\section{Übungen und Folien}
\begin{frame}{\TeX nische Umsetzung: Übungszettel}
• gewünscht: einheitliches Layout für alle Übungen
• eigene Klasse zum Setzen der Übungen und Lösungen
• da die Übungen auch im Quellcode online stehen: getrennte Dateien für Übungen und Lösungen!
•[⇒] Klassenoption zum Einbinden der Lösungen
• automatisiertes Finden zusammengehöriger Übungs-Lösungs-Blöcke
\•
\end{frame}

\begin{frame}{Übungsbetrieb}
• Übungen werden Freitag nach der Vorlesung online gestellt
• Bearbeitungszeit: eine Woche
• Abgabe vor der nächsten Vorlesung
• Korrekturzeit: eine Woche
• Besprechung in der übernächsten Vorlesung
\• 
\end{frame}

\begin{frame}{Bewertung}
• Aus Zeitgründen: HiWi
• Abzüge für: fehlende Übungsteile
•[] Code, der \TeX-Fehler verursacht
•[] unschöne, unelegante Lösungen
•[] nicht lauffähiger Code (wg. Kodierung o.\,ä.)
• Anmerkungen zu:
•[] überflüssigem, unnötigem Code
•[] Rechtschreibfehler
•[] besserer Codestil (Einrücken, Strukturieren, …) 
\•
\end{frame}

\end{document}